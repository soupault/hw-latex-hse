% Этот шаблон документа разработан в 2014 году
% Данилом Фёдоровых (danil@fedorovykh.ru) 
% для использования в курсе 
% <<Документы и презентации в LaTeX>>, записанном НИУ ВШЭ
% для Coursera.org: http://coursera.org/course/latex .
% Исходная версия шаблона --- 
% https://www.writelatex.com/coursera/latex/1.2

\documentclass[a4paper,12pt]{article} % добавить leqno в [] для нумерации слева

%%% Работа с русским языком
\usepackage{cmap}					% поиск в PDF
\usepackage{mathtext} 				% русские буквы в формулах
\usepackage[T2A]{fontenc}			% кодировка
\usepackage[utf8]{inputenc}			% кодировка исходного текста
\usepackage[english,russian]{babel}	% локализация и переносы

%%% Дополнительная работа с математикой
\usepackage{amsmath,amsfonts,amssymb,amsthm,mathtools} % AMS
\usepackage{icomma} % "Умная" запятая: $0,2$ --- число, $0, 2$ --- перечисление

%% Номера формул
\mathtoolsset{showonlyrefs=true} % Показывать номера только у тех формул, на которые есть \eqref{} в тексте.

%% Шрифты
\usepackage{euscript}	 % Шрифт Евклид
\usepackage{mathrsfs} % Красивый матшрифт

%% Свои команды
\DeclareMathOperator{\sgn}{\mathop{sgn}}

%% Перенос знаков в формулах (по Львовскому)
\newcommand*{\hm}[1]{#1\nobreak\discretionary{}
{\hbox{$\mathsurround=0pt #1$}}{}}

%%% Заголовок
\author{\LaTeX{} в Вышке}
\title{1.2 Математика в \LaTeX}
\date{\today}

\begin{document} % конец преамбулы, начало документа

\maketitle

Первый         абзац.

Второй абзац.
$2 +    2 =   4   $. Текст абзаца.
\[  2+2=4  \]

$2,4$, $(2, 4)$

Текст текст текст текст текст текст текст текст текст текст текст $1\hm{+}2+3+4+5+6=21$

\begin{equation}\label{eq:mrmc}
MR=MC
\end{equation}


\eqref{eq:mrmc}  на стр. \pageref{eq:mrmc} --- условие максимизации прибыли.


\section{Нюансы работы с формулами}

\subsection{Дроби}

\[\frac{1+\dfrac{4}{2}}{6} = 0,5\]

\subsection{Скобки}

\[ \left(2+\frac{9}{3}\right) \times 5 = 25 \]

\[  [2+3]  \]

\[ \{2+3\}  \]

\subsection{Стандартные функции}

$\sin x = 0$, $\cos x = 1$, $\ln x = 5$

$\sgn  x = 1$

\subsection{Символы}

$2\times 2 \ne 5$

$A \cap B$, $A \cup B$

\subsection{Диакритические знаки}

$\overline{456789xyz}=5$, $\widetilde{eurhkjs7} = 8$

\subsection{Буквы других алфавитов}

$\tg \Phi = 1$

$\epsilon$, $\phi$

$\varepsilon$, $\varphi$

\section{Формулы в несколько строк}

\subsection{Очень длинная формула}

\begin{multline}
	1+ 2+3+4+5+6+7+\dots + \\ 
	+ 50+51+52+53+54+55+56+57 + \dots + \\ 
	+ 96+97+98+99+100=5050 \tag{S} \label{eq:sum}
\end{multline}


\subsection{Несколько формул}

\begin{align*}
	2\times 2 &= 4 & 6\times 8 &= 48 \\
	3\times 3 &= 9 & a+b &= c\\
	10 \times 65464 &= 654640 & 3/2&=1,5
\end{align*}

\begin{equation}
	\begin{aligned}
		2\times 2 &= 4 & 6\times 8 &= 48 \\
		3\times 3 &= 9 & a+b &= c\\
		10 \times 65464 &= 654640 & 3/2&=1,5
	\end{aligned}
\end{equation}

\subsection{Системы уравнений}

\[
	\left\{
		\begin{aligned}
			2\times x &= 4  \\
			3\times y &= 9\\
			10 \times 65464 &= z\\
		\end{aligned}
	\right.
\]

\[
	|x|=\begin{cases}
		x, &\text{если }  x \ge 0 \\
		-x, &\text{если } x<0
	\end{cases}
\]

\section{Матрицы}

\[
	\begin{pmatrix}
		a_{11} & a_{12} & a_{13} \\
		a_{21} & a_{22} & a_{23}
	\end{pmatrix}
\]

\[
	\begin{vmatrix}
		a_{11} & a_{12} & a_{13} \\
		a_{21} & a_{22} & a_{23}
	\end{vmatrix}
\]

\[
	\begin{bmatrix}
		a_{11} & a_{12} & a_{13} \\
		a_{21} & a_{22} & a_{23}
	\end{bmatrix}
\]

В уравнении \eqref{eq:sum} на стр. \pageref{eq:sum} много слагаемых.

\end{document} % конец документа

